%\subsection{Figures}
\newcounter{loopcntr}
\newcommand{\rpt}[2][1]{%
  \forloop{loopcntr}{0}{\value{loopcntr}<#1}{#2}%
}
\newcommand{\on}[1][1]{
  \forloop{loopcntr}{0}{\value{loopcntr}<#1}{&\cellcolor{blue}}
}
\newcommand{\off}[1][1]{
  \forloop{loopcntr}{0}{\value{loopcntr}<#1}{&}
}
\newcommand{\onx}[1][1]{
  \forloop{loopcntr}{0}{\value{loopcntr}<#1}{&\cellcolor{cyan}}
}

\label{figures2}
\begin{frame}\frametitle{Work Plan}
  \noindent\begin{tabular}{p{0.08\textwidth}*{16}{|p{0.019\textwidth}}|}
  % The top line
  Task
             & \multicolumn{7}{c|}{2018}
             & \multicolumn{9}{c|}{2019}\\
  % The second line, with its five years of four quarters
  % \rpt[2]{& 1 & 2 & 3 & 4} \\
  & 06 & 07 & 08 & 09 & 10 & 11 & 12
  & 01 & 02 & 03 & 04 & 05 & 06 & 07 & 08 & 09\\

  \hline
  % using the on macro to fill in twenty cells as `on'

  A  \on[3] \off[13] \\
  \hline
  B \on[5] \off[11]\\
  \hline
  C \off[5] \on[4] \off[7] \\
  \hline
  D  \off[6] \on[4] \off[6]\\
  \hline
  \end{tabular}
  \begin{enumerate}[A]
  \item Modular linearity with Capabilities
  \item Semantic approach to modular linearity
  \item Paxos \& Mungo
  \item Thesis write-up
  \end{enumerate}
\end{frame}
