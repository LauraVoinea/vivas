\paragraph{Operational Semantics.}
\label{subsec:os}
%
The operational semantics are defined as a reduction relation
on networks with the use of a standard structural congruence.
%
	Structural congruence
	is defined to be the smallest congruence relation satisfying the following rules:

	$
	\begin{array}{c}
		B \pll B' \equiv B' \pll B
		\andalso \andalso
		(B \pll B') \pll B'' \equiv B \pll (B' \pll B'')
		\\
		N_1 \npll N_2 \equiv N_2 \npll N_1
		\andalso \andalso
		(N_1 \npll N_2) \npll N_3 \equiv N_1 \npll (N_2 \npll N_3)
		\andalso \andalso
		N \equiv \net{\nil \recover \nil} \npll N
		\\
		\newn{n} N \npll M \equiv \newnp{n}{N \npll M} \quad \IF n \notin \fn{M}
		\\
		\net{P \recover R \pll B} \equiv \net{P' \recover R' \pll B'} \quad \IF P \equiv P' \AND R \equiv R' \AND B \equiv B'
	\end{array}
	$
%
Parallel composition is commutative and associative for buffer terms
and for network terms, with the $\netnil$ as the unit
for network terms.
Scope extrusion holds for networks.
The last clause extends congruence for network nodes.

\begin{figure}[!h]
	\[
%	\small
%	\def\arraystretch{1.3}
	\begin{array}{c}
		\tree {
			%\forall i \in I, (l,l_i) \in G
		}{
			\net{\request{a}{\aggr x} P \recover R \pll B} \,\npll\,
			\Par{i}{I}{\net{\accept{a}{x} Q_i \recover R_i \pll B_i}}
			\\
			\qquad\quad
			\reduces
			\newnp{s}{\net{P \subst{\aggr s}{\aggr x} \recover R \pll B \pll \squeue{\aggr s}{0}{\equeue} } \npll
		    \Par{i}{I}{\net{Q_i \subst{s}{x} \recover R_i \pll B_i \pll \squeue{s}{0}{\equeue} }}}
		}{\Connect}
%
		\treeline
%
		\tree {
			%\forall i \in I, (l, l_i) \in G
		}{
			\net{\pout{\aggr s}{e} P \recover R \pll B \pll \squeue{\aggr s}{c}{\pol{m}} } \,\npll\,
			\Par{i}{I}{\net{P_i \recover R_i \pll B_i \pll \squeue{s}{c}{\pol{m}_i}}}
			\\
			\qquad\qquad 
			\reduces
			\net{P \recover R \pll B \pll \squeue{\aggr s}{c + 1}{\pol{m}}} \,\npll\,
			\Par{i}{I}{\net{P_i \recover R_i \pll B_i \pll \squeue{s}{c + 1}{\pol{m}_i \cat e} } }
		}{\Broadcast}
%
		\treeline
%
		\tree {
			%(l_1,l_2) \in G \andalso
			c_1 \geq c_2
		}{
			\net{\pout{s}{e} P_1 \recover R_1 \pll B_1 \pll \squeue{s}{c_1}{\pol{m}}} \,\npll\,
			\net{P_2 \recover R_2 \pll B_2 \pll \squeue{\aggr s}{c_2}{\pol{h}} }
			\\
			\qquad\qquad
			\reduces
			\net{P_1 \recover R_1 \pll B_1 \pll \squeue{s}{c_1 + 1}{\pol{m}} } \,\npll\,
			\net{P_2 \recover R_2 \pll \squeue{\aggr s}{c_2}{\pol{h} \cat (c_1, e)} }
		}{\Unicast}
%
		\\[6mm]
%
		\tree {
		}{
			\net{\pin{s}{x} P \recover R \pll B \pll \squeue{s}{c}{e \cat \pol{m}}}
			\reduces
			\net{P \subst{e}{x} \recover R \pll B \pll \squeue{s}{c}{\pol{m}}}
		}{\Receive}
%
		\treeline
%
		\tree {
			\pol{h}' = \newbuffer{\pol{h}}{c + 1}
			\andalso
			e = \gthr{\pol{h}}{c + 1}
		}{
			\net{\pin{\aggr s}{x} P \recover R \pll B \pll \squeue{\aggr s}{c}{ \pol{h} }}
			\reduces
			\net{P \subst{e}{x} \recover R \pll B \pll \squeue{\aggr s}{c + 1}{ \pol{h}' }}
		}{\Gather}
%
		\treeline
%
		\tree {
			%\forall i \in I, (l, l_i) \in G
		}{
			\begin{array}{l}
				\net{\sel{\aggr s}{\ell} P \recover R \pll B \pll \squeue{\aggr s}{c}{\pol{m}}} \,\npll\,
				\Par{i}{I}{\net{P_i \recover R_i \pll B_i \pll \squeue{s}{c}{\pol{m}_i}}}
				\\
				\qquad\qquad 
				\reduces
				\net{P \recover R \pll B \pll \squeue{\aggr s}{c + 1}{\pol{m}}} \,\npll\,
				\Par{i}{I}{\net{P_i \recover R_i \pll B_i \pll \squeue{s}{c + 1}{\pol{m}_i \cat \ell} }}
			\end{array}
		}{\Select}
%
		\treeline
%
		\tree {
			k \in I
		}{
			\net{\branch{s}{\ell_i: P_i}_{i \in I} \recover R \pll B \pll \squeue{s}{c}{\ell_k \cat \pol{m}}}
			\reduces
			\net{P_k \recover R \pll B \pll \squeue{s}{c}{\pol{m}}}
		}{\Branch}
%
		\treeline
%
		\tree[\Loss] {}{
			\net{\pout{s}{e} P \recover R \pll B \pll \squeue{s}{c}{\pol{m}}}
			\reduces
			\net{P \recover R \pll B \pll \squeue{s}{c + 1}{\pol{m}}}
		}{}
%
		\andalso
%
		\tree[\Recover] {}{
			\net{P \recover R \pll B}
			\reduces
			\net{R \recover R}
		}{}
%
		\\[5mm]
%
		\tree {
			\condtrue \econd
			\andalso
			\fs{B'} = \fs{P_2} - \fs{P_1}
		}{
			\net{\cond{\econd}{P_1}{P_2} \recover R \pll B \pll B'}
			\reduces
			\net{P_1 \recover R \pll B}
		}{\True}
%
		\treeline
%
		\tree {
			\not\condtrue \econd
			\andalso
			\fs{B'} = \fs{P_1} - \fs{P_2}
		}{
			\net{\cond{\econd}{P_1}{P_2} \recover R \pll B \pll B'}
			\reduces
			\net{P_2 \recover R \pll B}
		}{\False}
%
		\treeline
%
		\tree[\RPar] {
			N \reduces N'
		}{
			N \npll M \reduces N' \npll M
		}{}
		\andalso
		\tree[\RCong] {
			N \equiv N_1
			\andalso
			N_1 \reduces N_2
			\andalso
			N_2 \equiv N'
		}{
			N \reduces N'
		}{}
		\andalso
		\tree[\RRes] {
			N \reduces N'
		}{
			\newn{n} N
			\reduces
			\newn{n} N'
		}{}
	\end{array}
	\]
        \caption{Reduction rules for the asynchronous broadcast calculus.}
        \label{fig:reduction-relation}
\vspace{-5mm}
\end{figure}


The operational semantics is defined as the smallest relation on networks,
$N \reduces N'$, satisfying the rules given in \Figure{reduction-relation}.
%
Rule \Connect establishes a session between a request network node and
several accept network nodes. It is a broadcast (one-to-many) communication,
where there might not be any accepting nodes ($I = \emptyset$).
%
The interaction creates a fresh session $s$ with a unique $\aggr s$-endpoint
in the request side and a shared $s$-endpoint on the accept sides.
A corresponding session buffer term is created in all connected network nodes.

There are two kind of session communication distinguished
by the interaction between $\aggr s$-endpoint and $s$-endpoint:
broadcast communication; and unicast communication.
%
Rule \Broadcast defines the asynchronous broadcasting semantics;
the $\aggr s$-endpoint broadcasts message $e$, with $e$
being enqueued to the buffers of the $s$-endpoints,
thus modelling asynchrony.
%
The crucial condition for broadcast interaction is that
all participating nodes exist, i.e.~are synchronised,
in the same session state $c$. After the interaction all
participating nodes update their state to $c+1$.
%
Unreliability is modelled by the fact that set $I$ is chosen
arbitrarily, possibly being empty, even if there is a
node available for communication.

Rule \Unicast defines the case where an $s$-endpoint enqueues a
message $e$ in the $\aggr s$-endpoint buffer.
Asynchronous communication requires that an $s$-endpoint may
interact with the $\aggr s$-endpoint buffer if it exists in the same or in a
later state than the session state of the $\aggr s$-endpoint -- condition $c_1 \geq c_2$.
%
Message $e$ when enqueued is tagged with the sender's session
state $c$, as $h = (c, e)$, since all the messages on the same
state will be gathered by the $\aggr s$-endpoint (rule \Gather).

Rule \Receive defines the interaction of a process with its $s$-session
buffer; a \Rcv process on the $s$-endpoint receives the next available
expression, $e$, from the $s$-session buffer. Session state $c$ is not updated,
because it was updated by the operation that stored expression $e$ in the buffer.
%
Rule \Gather defines the interaction between a process and its $\aggr s$-session
buffer, where it gathers, via the \mult operator, all the expression messages that
are tagged with state, $c$, of the state of the $\aggr s$-endpoint. After
the reduction the state of the $\aggr s$-endpoint increases by $1$.
%
The rule uses auxiliary operations \gthr{\pol{h}}{c} and \newbuffer{\pol{h}}{c}:

\noindent
$
\small
\arraycolsep=2pt
\begin{array}{rclcrclcrclr}
	\gthr{(c, e) \cat \pol{h}}{c} &=& e \mult \gthr{\pol{h}}{c}
	& &
	\gthr{\equeue}{c} &=& \unit
	& &
	\gthr{(c', e) \cat \pol{h}}{c} &=& \gthr{\pol{h}}{c} & \IF c' \not= c
	\\
	\newbuffer{(c, e) \cat \pol{h}}{c} &=& \newbuffer{\pol{h}}{c}
	&\ &
	\newbuffer{\equeue}{c} &=& \equeue
	& &
	\newbuffer{(c', e) \cat \pol{h}}{c} &=& (c', e) \cat \newbuffer{\pol{h}}{c} & \IF c' \not= c
\end{array}
$

\noindent
Operation \gthr{\pol{h}}{c} goes through the messages $\pol{h}$ and returns,
up-to operator \mult, all expressions $e$ tagged with session state $c$, $(c, e)$.
Operation \newbuffer{\pol{h}}{c} returns a new $\pol{h}'$ by removing
all messages tagged with session state $c$, $(c, e)$.
%
The \Gather semantics also describes the case where the $\aggr s$-endpoint
gathers no messages, capturing the case where all message where either lost
or not delivered yet; operator \gthr{\pol{h}}{\pol{s}} will
return unit, \unit, if there are no messages tagged with state $c$ in \pol{h}.
%
The gather pattern is common in ad-hoc networks; see, for example, the RIME
communication stack~\cite{Dunkels2007} for wireless sensor networks.

Rules \Select and \Branch are similar to rules \Broadcast and \Receive;
the $\aggr s$-endpoint selects and broadcasts a label to the corresponding
$s$-endpoint buffer terms, and dually the $s$-endpoints receive a label
from its session buffer and proceeds accordingly.
%
The case where an $s$-endpoint is not defined since gathering multiple
labels on the $\aggr s$-endpoint makes no sense.

An implementation of the above semantics in real systems,
e.g.~wireless ad-hoc networks, requires that sent messages are
tagged with the session state of the sender.
This allows a perspective receiver to check the session state
conditions of the interaction and act accordingly,
e.g.~implementing a broadcast requires from a receiver to check its
session state against the message tag and only then store a broadcast message.

The semantics of the calculus offer flexibility, in a context
of asynchrony and unreliability.
%
Rule \Loss allows for message loss on the $s$-endpoint;
a process simply proceeds by dropping is sending prefix.
%
On a message loss the session state is updated/increased, allowing the $s$-endpoint
to synchronise with the $\aggr s$-endpoint, if the $\aggr s$-endpoint is
in a later session state, i.e.~a session state with a higher $c$ number.
%
The \Loss and \Unicast semantics are realistic since, e.g.~in sensor
networks, a sender cannot possibly know if a message was received by the
receiver and should always update its state.
%
Rule \Recover allows a network node to unconditionally recover
by replacing the running process with a recovery process $R$.
The recovery is hard: all active sessions and session buffers are dropped.
The purpose of the semantics is to recover from a situation where a process
may not be able to continue, due to unreliability.
The rules do not have a condition that triggers recovery:
this allows nodes to act autonomously, which means that
network nodes can recover even though their session state is still
synchronised, a situation that allows us to
model and react to real situations where a time-out
or failure detection occurs.
%
Rules \True and \False allow flexibility when handling shared session endpoints.
The conditional process gives the ability to discontinue using some
sessions in its branches. Thus, a session can be dropped based on the
condition evaluation.
%
When the process proceeds to a branch it must drop the buffers not used
by the continuation (condition $\fs{B'} = \fn{P_i} - \fn{P_j}, i \not= j$).
The case where no sessions are dropped, corresponds to standard conditional
semantics.
Dropping a communication session
is a common pattern in the context of
unreliable communication.

Finally, rules \RPar, \RRes, and \RCong are congruence for operators
$\npll$, name restriction, and structural congruence, respectively.
