\section{Work Undertaken}
\label{Research}
%
% In the past year I have largely hit the goals agreed upon at the
% last annual review, an overview of which I present here. Some items, remained on hold, while other research problems appeared.

Over the past year the focus of my research has shifted to a more theoretical one. As such, some of the goals stated last year have been met, while others have been superseded by new ones. An overview of which I present here.

\subsection{Modular linearity}
\label{sub:modlinearity}
Existing work typically assumes linear type systems as necessary for session types. In order to maintain session fidelity and ensure that all communication actions in a session type occur, session type systems typically require that endpoints are used linearly: each endpoint must be used exactly once.
This work sets out to investigate whether linear types are necessary to provide resource control/ownership for session types; what are the characteristics necessary to assure session fidelity; what other semantic properties should/might be guaranteed; how can more general approaches to resource control/ownership be integrated with session types.

Our approach builds on work done to deal with memory management, explicit region deallocation in the Capability Calculus~\cite{Crary:1999:TMM:292540.292564}; changing the type of a mutable object whenever the contents of the object is changed~\cite{Ahmed:2007:LLL:1365997.1366003}; or aliasing data structures that point to linear objects~\cite{FahndrichM:adofpl1}. The work presented below is inspired by the work done by Fähndrich and DeLine~\cite{FahndrichM:adofpl1}


\subsubsection{Capabilities in $\pi$-calculus with sessions}

% \section{Modular linearity}



% Existing work assumes linear type systems as necessary for session types.
% Open questions: are linear types necessary to provide resource control/ownership?  What characteristics are necessary to assure session fidelity?  What other semantic properties should/might be guaranteed? How can more general approaches to resource control/ownership be integrated with session types?
%Rather than check linearity at runtime check in some other way

% Existing work typically assumes linear type systems as necessary for session types. In order to maintain session fidelity and ensure that all communication actions in a session type occur, session type systems typically require that endpoints are used linearly: each endpoint must be used exactly once.

% We present a simple, but expressive type system that supports strong updates—updating a memory cell to hold values of unrelated types at different points in time. Our formulation is based upon a standard linear lambda calculus and, as a result, enjoys a simple semantic interpretation for types that is closely related to models for spatial logics. The typing interpretation is strong enough that, in spite of the fact that our core programming language supports shared, mutable references and cyclic graphs, every well-typed program terminates.
We present a simple type system based on $\pi$-calculus with session types and enriched with capabilities. Our approach builds on work done to deal with memory management, explicit region deallocation in the Capability Calculus~\cite{Crary:1999:TMM:292540.292564}; changing the type of a mutable object whenever the contents of the object is changed~\cite{Ahmed:2007:LLL:1365997.1366003}; or aliasing data structures that point to linear objects~\cite{FahndrichM:adofpl1}. The work presented below is particularly inspired by the work done by F\"ahndrich and DeLine~\cite{FahndrichM:adofpl1}. The formulation allows aliasing while ensuring through capabilities that all communication actions in a session type occur.


\begin{figure}[H]
  \centering
  $
  	\arraycolsep=1pt
  	\begin{array}{lclr}
    \text{Ordinary types } & T \DEF & \tvar int \OR \tvar Bool \\
    \\
    % \sigma \DEF \exists [q | \{q \gto h\}]. tr(q)
           % \OR T
    % \\
    % G \DEF q
    % \\

		\text{Session types} & S	\DEF &	\tout S U  & \text{output}\\
			& \OR	&	\tin S U & \text{input}\\
			& \OR	&	\tselI{\ell}{S} & \text{choice}\\
			& \OR	&	\tbranchI{\ell}{S} & \text{branch} \\
			& \OR	&	\tend & \text{terminated session}\\
			% \OR		\tvar Bool
      \\

    \text{Cababilities} & C \DEF & \econtext
       \OR  \{\rho \gto S \} \otimes C
       \OR  \epsilon \otimes C \\
    \\

    \text{Types} & \tau \DEF & tr(\rho) & \text{ tracked type} \\
         & \OR & T & \text{ ordinary types}\\
         & \OR & S &  \text{session types}\\
         & \OR & C & \text{capabilities}\\
      \\

    \text{Type Environments} & \Gamma \DEF & \econtext \bnfbar \Gamma, x: T  \\
         & \Delta \DEF &  \econtext \bnfbar \rho,\Delta \bnfbar \epsilon,\Delta\\
      \\
    \end{array}
	$
% \end{definition}
\caption{Types}\label{types}
\end{figure}
The type language distinguishes several kinds of types. Ordinary types, ranged over by T, are integers and booleans. The syntax of session types is standard.  Let S, U range over types. A type can be end, the type of the terminated channel where no communication can take place any longer. $\tout S U$ and $\tin S U$ are types for sending or receiving a value of type S with continuation of type U.  $\tselI{\ell}{S}$ and $ \tbranchI{\ell}{S}$ are sets of labelled types indicating, respectively, internal and external choice. The labels are all different and the order of the labelled types does not matter. Capabilities, ranged over by C map a channel name to a session type S. When communication starts, a new capability is added to the set, when the communication ends, it is removed. The types $\tau$ range over ordinary types, session types, capabilities, and tracked types. Tracked types are singleton types, $tr(\rho)$, where the key $\rho$ is a static name for the channel being tracked. We separate the handle to a linear object, $tr(\rho)$, from the capability to access that type.

The conjunction of capabilities is formed via $C_1 \otimes C_2$, expressing that keys in $C_1$ are disjoint from keys in $C_2$.



\begin{figure}[H]
  \centering
  $
  	\arraycolsep=1pt
  	\begin{array}{rclr}
  		P, Q &\bnfis& \nil & \Inact
  		\\
  		&\bnfbar& \pout{x}{v} P & \Send
      \\
      &\bnfbar& \pin{x}{y} P & \Rcv
      \\
  		& \bnfbar & \sel{x}{\ell_j} P & \Selection
      \\
      &\bnfbar& \branch{x}{\ell_i: P_i}_{i \in I} & \Branching
      \\
      &\bnfbar& P\pll Q & \Composition
  		\\
      &\bnfbar& \newn{xy} P& \Restriction
  		\\
      &\bnfbar& \cond{v}{P}{Q}& \Conditional
      \\
      v &\bnfis& x & \Variable
      \\
      &\bnfbar& \true \bnfbar \false & \Booleans
  	\end{array}
  $
% \end{definition}
\caption{Processes}\label{types}
\end{figure}

The syntax of processes is standard. Let P, Q range over processes, x, y over variables, v over values, and l over labels. Process $\nil$ is the terminated process. The send process $\pout{x}{v} P$ sends a value v on channel x and proceeds as process P; the receive process $\pin{x}{y} P$ receives a value on channel x, stores it in variable y and proceeds as P. The internal choice process $\sel{x}{\ell_j} P $selects label $\ell_j$ on channel x and proceeds as P. The branching process $\branch{x}{\ell_i: P_i}_{i \in I}$ offers a range of labelled alternatives on channel x, followed by their respective process continuation.  The order of labelled processes is not important and the labels are all different. Process $\cond{v}{P}{Q}$ proceeds as either P or Q depending on the value of boolean v. Process $\cond{v}{P}{Q}$ is the parallel composition of processes P and Q. Process $\newn{xy} P$ restricts variables x, y with scope P. It states that variables x and y are bound with scope P, and most importantly, are bound together, by representing two endpoints of the same channel. When occurring under the same restriction, x and y are called co-variables.


Typing judgments have the form $\Delta; \Gamma \types P; C$.
$\Gamma$ maps program variables to nonlinear types $T$ only. Linearity is enforced via capabilities, rather than via environment splitting as is the case in the standard $\pi$-calculus with sessions. A process is either correctly typed or not; we do not assign types to processes.

% \begin{definition}[Typing Rules]
%   \label{def:linear_typing}
%   \vspace{-2mm}
\begin{figure}[H]
\centering

  \[
  \begin{array}{cc}
  \tree{
  \\
  }{
    \Delta; \Gamma , x: T \types x: T
  }{}{\TVar}
  % &
  \\ \\
  \qquad
  \tree{
  \Delta; \Gamma
  \andalso
    v \in T
  }{
    \Delta; \Gamma \types v: T, C
  }{}{\TVal}

  % &
  % \qquad
  \\ \\
  \tree{
  \\
  }{
    \Delta; \Gamma , x: tr(\rho_x) \types x: tr(\rho_x); \{\rho_x \mapsto S\}
  }{}{\TVar}
  \\ \\
  \tree{
  C \text{ completed}
  }{
  \Delta; \Gamma \types \nil; C
  }{\TInact}
  % &
  % \qquad
  \\ \\
  \tree{
  \Delta; \Gamma \types P; C_P
  \andalso
  \Delta; \Gamma \types Q; C_Q
  }{
    \Delta; \Gamma \types P \pll Q; C_P \otimes C_Q
  }{}{\TPar}
  \\ \\
   \tree{
     \Delta; \Gamma, x: tr(\rho_x), y: tr(\rho_y) \types P; C \otimes \{\rho_x \mapsto S\} \otimes \{\rho_y \mapsto \tdual S\}
   }{
     \Delta; \Gamma \types \newn{xy} P; C_P
   }{}{\TRes}
   \\ \\
   % &
   % \qquad
    \tree{
      \Delta; \Gamma \types x: Bool \andalso \Delta; \Gamma \types P; C \andalso \Delta; \Gamma \types Q; C
    }{
      \Delta; \Gamma \types \cond{x}{P}{Q}; C
    }{}{\TCond}
    \\ \\
    \tree{
    \Delta, \rho_y; \Gamma, x : tr(\rho_x), y: tr(\rho_y) \types P; C \otimes \{\rho_x \mapsto U\} \otimes \{\rho_y \mapsto S\}
      }{
        \Delta; \Gamma, x: tr(\rho_x) \types \pin{x}{y} P; C \otimes \{\rho_x \mapsto \forall [\rho_y].\tin S U \}
      }{\TRcv}

      \\ \\
      \tree{
      \Delta; \Gamma, x : tr(\rho_x), y: T \types P; C \otimes \{\rho_x \mapsto U\}
        }{
          \Delta; \Gamma, x: tr(\rho_x) \types \pin{x}{y} P; C \otimes \{\rho_x \mapsto \tin T U \}
        }{\TRcvVal}
    % &
    % \qquad
    \\ \\
    \tree{
    \Delta, \rho_v; \Gamma, x: tr(\rho_x) \types P; C \otimes \{ \rho_x \mapsto U\}
      \andalso
    \Delta; \Gamma \types v : tr(\rho_v); \{\rho_v \mapsto S\}
    }{
    \Delta; \Gamma, x: tr(\rho_x) \types \pout{x}{v} P; C \otimes \{\rho_x \mapsto  \forall [\rho_v].\tout S U\} \otimes \{\rho_v \mapsto S\}
    }{\TSnd}
    \\ \\
    \tree{
    \Delta; \Gamma, x: tr(\rho_x) \types P; C \otimes \{ \rho_x \mapsto U\}
      \andalso
    \Delta; \Gamma \types v : T
    }{
    \Delta; \Gamma, x: tr(\rho_x) \types \pout{x}{v} P; C \otimes \{\rho_x \mapsto  \tout T U\}
      }{\TSndVal}
  \\ \\
    \tree{
    \Delta; \Gamma, x: tr(\rho_x) \types P; C \otimes \{\rho_x \mapsto S_j\}, j \in I
    }{
    \Delta; \Gamma, x : tr(\rho_x) \types \sel{x}{\ell_j} P; C \otimes \{\rho_x \mapsto \tselI{\ell}{S}\}
    }{\TSel}
    %
    % &
    % \qquad
    % \andalso
    %
    \\ \\
    \tree {
    \Delta; \Gamma, x : tr(\rho_x) \types P_i; C \otimes \{\rho_x \mapsto S_i\}, \forall i \in I
    }{
      \Delta; \Gamma, x : tr(\rho_x) \types \branch{x}{\ell_i:P_i}_{i \in I}; C \otimes \{\rho_x \mapsto \tbranchOn{\ell_i:S_i}_{i \in I}\}
    }{\TBr}
  \end{array}
  \]

\caption{Typing rules}\label{typing_rules}
\end{figure}

Rule $\TVar$ states that a variable x is of type T, if this is the type assumed in the typing context. Rule $\TVal$ states that a value v, being either true or false, is of type Bool. $\TInact$ states that the terminated process $\nil$ is always well-typed. $\TPar$ types the parallel composition of two processes, using the split operator for typing contexts $\sessionop$ which ensures that each linearly-typed channel x, is used linearly, i.e., in $P \pll Q$, x occurs either in P or in Q but never in both. Rule $\TRes$ states that $\newn{xy} P$ is well typed if P is well typed and the co-variables have dual types, namely T and $\tdual T$. Rule $\TCond$ states that the conditional statement is well typed if its guard is typed by a boolean type and the branches are well typed under the same typing context. Rules $\TRcv$ and $\TSnd$ type the receiving and the sending of a value. Rule $\TBr$ types an external choice on channel x, checking that each branch continuation $P_i$ follows the respective continuation type of x. Dually, rule $\TSel$ types an internal choice communicated on channel x, checking that the chosen label is among the ones offered by the receiver and that the continuation proceeds as expected by the type of x.

The operational semantics of the language is standard, and can be seen in the appendix.

Proofs that this system still offers the same guarantees as the standard $pi$-calculus with session, namely proofs of progress and type preservation, are under way.

An implementation of this work is planned for later in the year. It will be implemented within the Mungo tool\cite{kouzapas16}, developed at Glasgow University. This will allow for more flexible aliasing within Mungo.


% Session types without tiers -- focused on exceptions, functional setting rather than process setting


An implementation of this work is planned for later in the year. It will be implemented within the Mungo tool\cite{kouzapas16}, developed at Glasgow University. This will allow for more flexible aliasing.

\subsubsection{A semantic approach to $\pi$-calculus with sessions}
To explore what characteristics are necessary to assure session fidelity or what other semantic properties might be guaranteed; we have taken a semantic approach in the style of \cite{Ahmed:2004:STM:1037736}. This is achieved through defining logical relations that explain the meaning of the type in terms of the operational semantics of the language. Using this model of types, I prove each typing rule as a lemma.

Logical relations have been used in the works of P{\'e}rez et al. in~\cite{  10.1007/978-3-642-28869-2_27} and in~\cite{PEREZ2014254} for proving properties about session types in an intuitionistic linear logic setting. However, this approach remains largely unexplored in the context of session types.
In particular, we want to analyse properties of the type system, especially linearity and how it is used to guarantee session safety, separately from the format of the typing rules.

This work has been started fairly recently, with definitions being under construction.


 % The idea is to be able to analyse properties of the type system, especially linearity and how it is used to guarantee session safety, separately from the format of the typing rules.

% For each type in the language, I define logical relations that explain the meaning of the type in terms of the oper- ational semantics of the language. Using this model of types, I prove each typing rule as a lemma.

%
% Type systems—and the associated concept of “type soundness”—are one of the biggest success stories of foundational PL research. Originally proposed by Robin Milner in 1978, type soundness asserts that well-typed programs can’t “go wrong” (i.e., exhibit undefined behaviors), and it is widely viewed as the canonical theorem one must prove to establish that a type system is doing its job. In the early 1990s, Wright and Felleisen introduced a simple syntactic approach to proving type soundness, which was subsequently popularized as the method of “progress and preservation” and has had a huge impact on the study and teaching of PL foundations. Many research papers that propose new type systems conclude with a triumphant statement of syntactic type soundness, and for many students it is the only thing they learn to prove about a type system.
%
% Unfortunately, syntactic type soundness is a rather weak theorem. First of all, its premise is too strong for many practical purposes. It only applies to programs that are completely well-typed, and thus tells us nothing about the many programs written in “safe” languages that make use of “unsafe” language features. Even worse, it tells us nothing about whether type systems achieve one of their main goals: enforcement of data abstraction. One can easily define a language that enjoys syntactic soundness and yet fails to support even the most basic modular reasoning principles for closures, objects, and ADTs.
%
% In this talk, I argue that we should no longer be satisfied with just proving syntactic type soundness, and should instead start proving a stronger theorem—semantic type soundness—that captures more accurately what type systems are actually good for. In a semantic soundness proof, one defines a semantic model of types as predicates on values, and then verifies the soundness of typing rules as lemmas about the model. By explaining directly what types “mean”, the semantic approach to type soundness is a lot more informative than the syntactic one. In particular, it can serve to establish what data abstraction guarantees a language provides, as well as what it means for uses of unsafe language features to be “safely encapsulated”.
%
% Semantic type soundness is a very old idea—Milner’s original formulation of type soundness was a semantic one—but it fell out of favor in the 1990s due to limitations and complexities of denotational models. In the succeeding decades, such limitations have been overcome and complexities tamed, via proof techniques that work directly over operational semantics. Thanks to the development of step-indexed Kripke logical relations, we can now scale semantic soundness to handle real languages, and thanks to advances in higher-order concurrent separation logic, we can now build (machine-checked) semantic soundness proofs at a much higher level of abstraction than was previously possible. The resulting “logical” approach to semantic type soundness yields proofs that are demonstrably more useful than their syntactic counterparts, and more fun as well.



\subsection{A Paxos implementation}

Paxos is a protocol for solving consensus in a network of unreliable processes. It ensures that a single value among the proposed values can be chosen. It assumes an asynchronous, non-Byzantine model.\cite{lamport1998part}

Two major advantages of Paxos are that it is provably correct in asynchronous networks that eventually become synchronous and it does not block if a majority of participants are available. Furthermore, it has provably minimal message delays in the best case. Despite its reputation of being difficult to understand \& implement it is widely used in systems such as Google's Chubby\cite{chandra2007paxos} or Yahoo's ZooKeeper. The protocol comes with three roles and a two-phase approach. A proposer responsible for initiating the protocol, that handles client requests and
proposes values to be chosen. An acceptor that responds to messages from proposers by either rejecting them or agreeing in principle and making a promise about the proposals it will accept in the future. An a listener or learner, who wants to know which value was chosen. Each Paxos server can act as any or all 3 roles.\cite{lamport2001paxos} Some representations of the algorithm have been attempted using Scribble, combined with StMungo and Mungo to give a working Java code. However in trying to represent it some shortcomings of the toolset became apparent: representing broadcasting, representing quorum/a majority, representing express the dynamic aspects such as processes failing, restarting, expressing multiple instances of the protocol

Since, a successful implementation of Paxos has been achieved using the session type system for unreliable broadcast communication devised by Gutkovas, Kouzapas and Gay. The example has been written up as part of a paper describing the type system and will be submitted to the Logical Methods in Computer Science journal\footnote{https://lmcs.episciences.org}.

I will first briefly introduce the language I have used, followed by the implementation itself.
\subsubsection{Implementation language}

\paragraph{Syntax.}
\label{subsec:syntax}
%
Assume the following disjoint countable sets of names: $\mathcal C$ is the set of \emph{shared channels} ranged over by $a,b,c,\dots$; $\mathcal S$ is the set of \emph{session channels} ranged over by $s,s',\dots$, where each channel has two distinct endpoints $s$ and $\aggr s$
(we write $\kappa$ to denote either $s$ or $\aggr s$);
$\mathcal V$ is the set of \emph{variables} ranged
over by $x,y,z,\dots$;
and $\mathsf{Lab}$ is the set of labels ranged over
by $\ell, \ell', \dots$.
%
We write $k$ to denote either $\kappa$ or $x$ or $\aggr x$,
where $\aggr x$ is used to distinguish a variable used as a
$\aggr s$-endpoint.

Let $\mathcal E$ be a non-empty set of {\em expressions}
ranged over by $e,e',\dots$. Assume a binary operation
$\mult$ on $\mathcal E$ called {\em aggregation},
and element $\unit \in \mathcal E$ called {\em unit}.
%
Let $\mathcal F \subseteq \mathcal E$ be a non-empty set
of {\em conditions} ranged over by \econd, and assume a
truth predicate $\condtrue\, \subseteq \mathcal F$.
%
Elements of $\mathcal E$ %and $\mathcal F$
may contain variables.
% and the set of free variables is denoted by
%$\fn{e}$ and $\fn{\econd}$, respectively.
%
Assume a substitution function $e \subst{e'}{x}$.
%defined on $\mathcal E$ and $\mathcal F$
%such that
%$\fn{e\subst{e'}{x}} \subseteq \fn{e} \cup \fn{e'}$.
%and similarly for conditions.
%

The syntax of processes is then defined as:

%
\noindent
$
%	\arraycolsep=1pt
	\begin{array}{rclrclrclr}
		P &\bnfis& \nil & \Inact &\bnfbar& \request{a}{\aggr x} P & \Request &\bnfbar& \accept{a}{x}{P} & \Accept
		\\
		&\bnfbar& \pout{k}{e} P & \Send &\bnfbar& \pin{k}{x} P & \Rcv %&\bnfbar& \ndet{P}{P} & \NDet
		& \bnfbar & \sel{k}{\ell} P & \Selection
		\\
		&\bnfbar& \branch{k}{\ell_i: P_i}_{i \in I} & \Branching %&\bnfbar& \appl{D}{\tilde{u}} & \PVar
%		&\bnfbar& \multicolumn{4}{l}{\Def{\set{\abs{D_i}{\tilde{x}_i} \defeq P_i}_{i \in I}}{P}} & \Recursion
%		\\
		& \bnfbar & \multicolumn{4}{l}{\cond{\econd}{P}{P}} & \Conditional
	\end{array}
$
%\vspace{-1mm}

% Variable and name binders are standard, so is $\fn{P}$.
%We also define $\fn{P}$ in the expected way.
%Terms \Request and \Accept bind $s$ in $P$, and term \Rcv binds $x$ in $P$.
%In term \Recursion $\abs{D}{\tilde{x}} \defeq P$ binds
%$\tilde{x}$ in $P$.
%We define $\fn{P}$ to be the set of free channels, session channels,
%and variables of $P$ in the standard way.
%We identify processes up to $\alpha$-equivalence.
%%
%

\Inact is the inactive term. \Request and \Accept express the
terms that are ready to initiate a fresh session on a shared channel $a$
via a request/accept interaction, respectively.
%
\Send defines a prefix ready to send an expression
on $k$, whereas term \Rcv defines a prefix ready to receive
a message on $k$ and substitute it on $x$.
%
Session selection is defined by term \Selection where
the prefix sends a label $\ell$ over session $k$.
Dually, session branching is defined by term \Branching
where the prefix receives a label from a predefined
set of label $\set{\ell_i}_{i \in I}$.

To introduce networks, assume a countable set of {\em networks}
$\mathcal N$ ranged over by $N$. Let symbol $m$, called
{\em message}, range over by expressions $e$ and labels $\ell$,
with $\tilde{m}$ a message vector. Also, let $h$ denote a message
tagged with a natural number $c$, $(c, e)$.
%

\noindent
$
%	\arraycolsep=1pt
	\begin{array}{rclrclrclr}
		B	&\bnfis&	\ebuffer & &\bnfbar& B \pll \squeue{s}{c}{\pol{m}}&  &\bnfbar& B \pll \squeue{\aggr s}{c}{\pol{h}} &
		\\
		N, M	&\bnfis&	\net{P \recover R \pll B} &\Node &\bnfbar& N \npll M &\Parallel &\bnfbar& \newn{n} N& \Restriction
	\end{array}
$

% We extend the $\fn{N}$ function to networks.
%
Term, $B$, is a parallel composition of session
buffers, that are used to store messages and to count
session state via natural number $c$.
Buffer terms are used to model a form of asynchrony that
preserves the order of received messages.
%
The purpose of counter $c$ is to keep track of the
session state in the presence of a lossy communication
and to synchronise the interaction between session prefixes.
%
Message loss in an unreliable setting leads to
participants that are not synchronised with the
overall protocol.
%
% To ensure correctness, many frameworks and algorithms
% that operate in an unreliable setting use message tagging
% or state counting; for example, the TCP/IP protocol tags
% packets with unique sequential numbers to maintain consistency
% in the case of packet loss.
%
% In this setting, state counting is necessary
% to maintain correct interaction within a session.
% The session type system provides static guarantees for a session despite
% the dynamic nature of session reduction.

Buffer terms on $s$-endpoints store messages $m$, while
Buffer terms on $\aggr s$-endpoints store expression messages
tagged with a session counter, $h = (c, e)$.
%
The session counter in $h$ distinguishes the session
state at which the expression $e$ was be received.
%
Network \Node consists of a process $P$,
%executing in location~$l$ (cf.~\cite{Hennessy2002}),
a recovery process $R$ that may take over if $P$
cannot proceed in a session, and the necessary buffer
terms used for asynchronous session communication.
%
A process may participate in several sessions, and therefore,
more than one buffer term may be present in a node.
The type system ensures that there is no more than one
buffer term on the same session in each network node.
We write $\net{P \recover R}$ for node
$\net{P \recover R \pll \ebuffer}$.

A network is a parallel composition of nodes --- term \Parallel.
We may write $\Par{i}{I}{N_i}$ for the parallel composition of
$N_1 \npll \cdots \npll N_n$
for (possibly empty) $I = \set{1, \dots, n}$. %, otherwise $\netnil$.
Network \Restriction binds both session and shared channels.

\begin{definition}[Session Type]
	Let $\mathcal B$ be a set of \emph{base types} ranged over by~$\beta$.
	Session types are inductively defined by the following grammar:


		\vspace{-2mm}
		\[
			T	\DEF	\tout \beta T
				\OR		\tin \beta T
				\OR		\tselI{\ell}{T}
				\OR		\tbranchI{\ell}{T}
				\OR		\tend
				\OR		\tvar t
				\OR		\trec t T
		\]

	\vspace{-2mm}
\end{definition}



%
%Type $\trec t T$ binds free occurrences of $t$ in $T$.
%We define a capture avoiding substitution on types $\subst{T}{t}$ in
%the usual way.
%We identify recursive types with their expansion: $\trec t T =
%T\subst{\trec t T}{t}$.
% The duality operator also follows the standard binary session type
% definition~\cite{Honda1998}.

We say that two types $T_1$ and $T_2$ are dual if $\tdual T_1 = T_2$.
Note $\tdual{\tdual{T}} = T$ for any $T$.

\noindent
$
	\arraycolsep=3pt
	\begin{array}{rclcrclcrclcrcl}
		\tdual\tend	&=&	\tend
		&&
		\tdual{\tout\beta T}	&=&	\tin\beta \tdual T
		&&
		\tdual{\tselI{\ell}{T}}	&=&	\tbranchI{\ell}{\tdual{T}}
		&&
		\tdual{\tvar t}			&=&	\tvar t
		\\
		&&&&
		\tdual{\tin\beta T}	&=&	\tout\beta \tdual T
		&&
		\tdual{\tbranchI{\ell}{T}} &=&	\tselI{\ell}{\tdual{T}}
		&&
		\tdual{\trec t T}	&=&	\trec t \tdual T
	\end{array}
$

%


%Next we define a message type syntax used
%to type message buffers.
%
% Message Types
% 	$
% 		M \bnfis \tempty \bnfbar \mtout. \beta M \bnfbar \msel \ell. M
% 	$
%
% Message types represent the sequence types
% of the values, $\mtout \beta. M$ or labels, $\msel \ell. M$
% in a session buffer term.
% %

\begin{definition}[Typing Context]
	We define $\Gamma$, $\Delta$, and $\Theta$ typing contexts:\\
	\begin{tabular}{l}
		$\Gamma \bnfis \econtext \bnfbar \Gamma, a: T \bnfbar \Gamma, x: \beta
		\qquad
		\Delta \bnfis \econtext \bnfbar \Delta, k: T \bnfbar \Delta, s: c
		$
		\\
		$\Theta \bnfis \econtext \bnfbar \Theta, \kappa: M \bnfbar \Theta, s: c$
	\end{tabular}
\end{definition}


%
Shared context $\Gamma$ tracks shared names used in a process,
while context $\Delta$ tracks linear (session) names or session variables
used in a process.
Context $\Theta$ is also linear and tracks the types of buffer terms.
%Both linear contexts track the state of a session.
We denote by $\Gamma, \Gamma'$ the concatenation of contexts $\Gamma$ and
$\Gamma'$, and similarly for $\Delta,\Delta'$ and $\Theta, \Theta'$.
%


%===============================================================================
\subsubsection{Implementation}

The session type representation of Paxos is used to check that implementations
correctly follow the protocol, rather than correctness of the protocol itself.
Session types can also help to identify subtle interactions such as branching or dropping sessions.
Furthermore, a session type representation allows the basic algorithm to be easily extended while
still providing formal guarantees.

Paxos agents implement various roles:
i) a \emph{proposer} initiates the paxos rounds,
in which it proposes a value to the acceptors;
ii) an \emph{acceptor} will accept a proposal if it is from the latest round i.e. with the highest round number.
A value accepted form a  majority of acceptors(quorum) signifies the reach of consensus and protocol termination; and
iii) a \emph{learner} is an agent that wants to know which value has been chosen.
%
The Paxos setting assumes lossy communication. Agents may crash and
recover.
%It is assumed that acceptors maintain persistent storage
%that survives crashes in which they record their intended responses
%before sending them~\cite{lamport2001paxos}.
Proposers record their highest round to stable storage and begin a new
round with a higher round number than previously used%~\cite{lamport2001paxos}.
If eventually a majority of the acceptor agents run for long enough without failing, consensus is guaranteed.

We implement the most basic protocol of the Paxos family, as introduced in~\cite{lamport2001paxos}.
The protocol proceeds over several rounds. A round has two phases, \PreparePh and \AcceptPh. Each run of the protocol decides on a single consensus value.

%\input{figures/fig-paxos}

The Paxos communication interaction is described by session type \PaxosType:

$
		\tout{\prep} \tin{\prom}
			\oplus \left\{
			\begin{array}{rl}
				\restart: & \tend,
				\\[0mm]
				\acceptPhase: & \tout{\messageTuple} \tin{\rnumber} \oplus \{ \restart: \tend, \chosen: \tend \}
			\end{array}
			\right\}
$

%
A Paxos agent is described by network node
$\PaxosNode_{n, v} = \net{\Paxos_{n, v} \recover \Paxos_{n, v}}$
with $n$ being the (fresh) number of the current protocol round, $v$
being the {\em consensus value} the agent currently holds and
process $\Paxos_{x, y}$ defined as:

\noindent
	$
		\begin{array}{rcl}
%			\multicolumn{3}{l}{\Paxos_{n, v} =}
%			\\
								&& \Def{
								\\
								&&
								\begin{array}{rcl}
									\ProposerDef{x, y}
									&\defeq&
										\request{a}{s}
										\pout{\aggr s}{x}
										\pin{\aggr s}{\set{(n, v)_i}_{i \in I}}
									\\
									&&	\If\ |\set{(n, v)_i)}_{i \in I}| \leq \frac{M}{2}\ \Then\ \sel{\aggr s}{\restart} \PaxosVar{x+1, y}\\
									&&	\Else\ \If\ \Max{\set{n_i}_{i \in I}} > x\ \Then\ \sel{\aggr s}{\restart} \PaxosVar{x+1, y}\\
									&&	\Else\ \sel{\aggr s}{\acceptPhase}
										\\ && \quad \pout{\aggr s}{x, v_h = \Max{\set{v \,|\, (n, v) \in \set{(n, v)_i}_{i \in I}}}}
										\pin{\aggr s}{\set{n_i}_{i \in I}}
										\\ && \quad \If\ |\set{n_i}_{i \in I}| \leq \frac{M}{2}\ \Then\ \sel{\aggr s}{\restart}\PaxosVar{x+1, y}
										\\ && \quad \Else\ \sel{\aggr s}{\chosen} \PaxosVar{x, v_h}
									\\[2mm]
									\AcceptorDef{x, y}
									&\defeq&
										\accept{a}{s}
										\pin{s}{x'}\ \pout{s}{x, y} (\AccVar{s, x, y}\ \Sum\ \accept{a}{s'} \pin{s'}{x''}
										\\ && \If\ (x'' > x')\  \Then\ \pout{s'}{x, y} \AccVar{s', x, y}\ \Else\ \AccVar{s, x, y})

									\\[2mm]
									\AccDef{w, x, y}
									&\defeq&
										w \triangleright
										\left\{
											\begin{array}{rl}
												\restart:	& \PaxosVar{x, y},
												\\[0mm]
												\acceptPhase:	& \pin{w}{x', y'} \pout{w}{x'}
														w \triangleright
														\left\{
														\begin{array}{rl}
															\restart: & \PaxosVar{x, y},
															\\[0mm]
															\chosen: & \PaxosVar{x', y'}
														\end{array}
														\right\}
											\end{array}
										\right\}
									\\[2mm]
									\PaxosDef{x, y}
									&\defeq&
										\ProposerVar{x, y} \Sum \AcceptorVar{x, y}
								\end{array}
						\\		&& }{\PaxosVar{n, v}}
		\end{array}
	$
%

Unlike ~\cite{lamport2001paxos}, our implementation allows a agent to non-deterministically behave either as a proposer,
(definition \ProposerDef{x, y}), or an acceptor
(definition \AcceptorDef{x, y}).

A proposer interacts within the same round with with multiple acceptors.
The communication behaviour of an Acceptor is described by the
session type \PaxosType, whereas the communication behaviour of the
proposer is described by the dual type $\tdual{\PaxosType}$.

During a round a Paxos agent may restart in which case it terminates its
current sessions and proceeds to the initial Paxos network,
$\Paxos_{x, y}$. Note that each time an initial Paxos agent enters
a new protocol round it establishes a new session.
Recovery by a Paxos agent is equivalent to a Paxos agent restart.

If a Paxos agent decides to behave as a proposer, it first requests
session communication and enters the \PreparePh phase. All Paxos
agents that accept a session request behave as acceptors.
%
The proposer then broadcasts a \textit{prepare},type \prep, request
with a fresh round (or session) number $n$.

All the acceptors that received the \textit{prepare} message reply
with a \textit{promise}, type \prom, not to respond to a prepare
message with a lower round number.
%
The promise message contains the current round number and the current
value of the acceptor. All the promises are gathered in a set by the
proposer, which then checks whether the majority of acceptors
have replied, i.e.\  $|\set{(n, v)_i)}_{i \in I}| \leq \frac{M}{2}$,
with $M$ the number of expected connected acceptors.

If the check fails the proposer sends a restart label to all the
acceptors, and restarts its own computation with an increment on its
round number, as in $\PaxosVar{x + 1, y}$. All acceptors that receive
label \restart restart.
%
If majority is achieved, the proposer checks whether any of the acceptors
has promised to reply on higher round numbers than $x$, in which case the
proposer increments its round number and restarts all agents within the
session via label \restart.

If both of these checks are passed, the protocol enters the \AcceptPh
phase. The proposer selects a value to submit to the acceptors
by inspecting the promises received and selecting the highest value,
i.e.~$v_h = \Max{\set{v \,|\, (n, v) \in \set{(n, v)_i}_{i \in I}}}$.
%
It then broadcasts an \textit{accept} message; selects label \acceptPhase
followed by the current round number together with the chosen highest value.
%
The acceptors reply with a message of type \rnumber, containing
their current round number. These messages are gathered by the
proposer and checked for majority,
in which case, consensus has been reached and the acceptors will
be informed via label \chosen. All informed agents
restart by updating their round number and consensus value.
Otherwise, if lack of majority is detected, the proposer increments
its round number and restarts all agents within the session via
label \restart.

On the acceptor side we use the non-deterministic construct to
capture the case of session initiations from multiple proposers.
In such a case, the session with the lowest round number is
dropped. A dropped session by an acceptor will have an impact to
the majority check by the corresponding proposer
of the session, thus checking for majority is crucial for reaching
consensus.

Following the semantics of our framework, messages can be arbitrarily
lost, thus triggering the recovery procedure for the corresponding
Paxos agent. Lost messages are taken into account by the logic of the
Paxos protocol and have an impact on the execution of the algorithm.
%
For example, lost messages lead to failure to reply by the majority of
acceptors, which implies failure to reach consensus within a round,
thus restarting a new protocol round.
%
%Furthermore, an Acceptor that becomes unsynchronised due to
%lost messages can recover and wait for a new session request
%to initiate a new protocol round.

We can type the $\PaxosNode_{n, v}$ node as:
$\Gamma, a: \PaxosType; \econtext \types \PaxosNode_{n, v}$.
%
Shared channel $a$ uses type \PaxosType, thus all establish sessions
follow the behaviour described by the \PaxosType session type.
%
We can then define a well-typed network, $N$, that runs the
Paxos protocol:
$
	N = \Par{i}{I}{\PaxosNode_{n_i, v_i}}
$
with $\Gamma, a: \PaxosType; \econtext \types N$. Typing is possible
due to the typing of network node $\PaxosNode_{n, v}$ and multiple
applications of rule \TPar.


An implementation of this work is planned for later in the year. It would be interesting to see whether any of the typing can be expressed/checked in Mungo, even if not all aspects are there.


\subsection{Other Activities}
\label{sec:Activities}

As part of the third year of PhD, various additional activities have been undertaken, such as attending conferences, ABCD group meetings of various sizes, seminars and talks(e.g. The Scottish programming language seminar series, FATA seminars) which improved my knowledge of the field and gave me some insight of the exciting research ongoing. Notably, I have attended the Oregon Programming Languages Summer School\footnote{https://www.cs.uoregon.edu/research/summerschool/summer17/}
which was a great opportunity to learn more about programming languages, from  foundational work on semantics and type theory to advanced program verification techniques; and had a big impact on the direction my research has taken since.

% I plan on attending the Oregon Programming Languages Summer School\footnote{https://www.cs.uoregon.edu/research/summerschool/summer17/} from 23 June to 8 July.
