% \begin{frame}\frametitle{Mungo}
%
% • Mungo is a Java front-end tool that statically checks the order of method calls.
% • It is based on the notions of typestate – describing non-uniform objects– and session types – describing a protocol on the method calls.
% • A Java class is augmented with a typestate. Mungo checks that method calls follow the declared typestate of an object.
% • Contributors: ABCD Glasgow team.
% • Link: http://www.dcs.gla.ac.uk/research/mungo/
% \end{itemize}
% \end{frame}

\begin{frame}\frametitle{What next?}
  \begin{itemize}
    \item Design and carry out experiments
    \item Investigate with H1 and H2 in various contexts
    \item Start with Java, popular mainstream language, and Mungo, Java based tool at Glasgow University
  \end{itemize}

  Mungo:
  \begin{itemize}
    \item Java front-end tool, developed at the University of Glasgow
    \item Statically checks the order of method calls
    \item A protocol or session type is represented as a separate typestate file, associated with a Java class.
    \item The protocol definition is described as a sequence of method calls, the order of which determines the validity of the protocol.

  \end{itemize}
\end{frame}


\begin{frame}[fragile]\frametitle{Proposed experiment}
% Null hypothesis: Usually a neutral statement stating that
% groups are not different (e.g., novices using a particular
% language will perform equally)
%
%
% Alternative hypothesis: Usually the opposite of the null
% hypothesis (e.g., novices using a particular language will
% perform better than another)
%
% Experimental design:
%
% Considering that the aim of this experiment is exploratory and the learning effect (performance improving with experience, even over very short periods of time) is of little concern in this case a within subjects design, every participant perform every task under every condition, will be used.
\textbf{Investigate how programmers work with Mungo.}\\
Tasks:
\begin{itemize}
  \item Modify an existing and debug a program (such as POP3);
  \item Starting from an English description code Iterator, Fibonacci, Two Buyer Protocol\\
\end{itemize}
Measurements:
\begin{itemize}
  \item time to complete each task, order in which tasks are performed, how many times the code is compiled, how many bugs/errors, is the end result correct, personal experience\\
\end{itemize}
Subjects:
\begin{itemize}
  \item 4th year computer science students\\
\end{itemize}
%
% the experimental design (e.g. AB experiments, repeated measures, Solomon 4-group designs),
% • the experimental unit (e.g. the concrete source code to be modified or to be written),
% • the measurement (e.g. time required to solve a problem),
% • the subjects (e.g. children, first year college students,
% professionals),
% • the sample size (e.g. 42 subjects)
% • the analysis technique (e.g. T-Test, structural equation modeling),
% • the effect size of the technique or the approach to be tested (so-called variance accounted for),
% • the conclusion step (e.g. alpha-level of .05), and
% • Threats to validity (e.g., non-standard experimental de-
% signs, missing data).

\end{frame}



  % Sample size:
  % Analysis technique:
%   Null hypothesis: Usually a neutral statement stating that
%   groups are not different (e.g., novices using a particular
%   language will perform equally)
%
%
%   Alternative hypothesis: Usually the opposite of the null
%   hypothesis (e.g., novices using a particular language will
%   perform better than another)
%
% Experimental design: AB experiment/Within-subjects designs:
% 20 minute tutorial on Session types
%
% Tasks:
% Measurements: time to complete each task, how many times the code is compiled, how many bugs/errors, is the end result correct
% Subjects: Computer scientists with Java experience
% Sample size:
% Analysis technique:
