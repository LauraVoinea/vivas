\section{Research Activities}
\label{sec:Research}

\subsection{Mungo}
\label{sub:Mungo}
Mungo\footnote{http://www.dcs.gla.ac.uk/research/mungo/} is a Java front-end tool, developed by Dr. Dimitrios Kouzapas at the University of Glasgow, used to statically check the order of method calls i.e. its typestate. It is implemented using the JastAdd framework\footnote{http://jastadd.org/web/extendj/}. A protocol or session type is represented as a separate typestate file, associated with a Java class.  The protocol definition is described as a sequence of method calls, the order of which determines the validity of the protocol.

Mungo checks that the object instantiating the class performs method calls as defined by its typestate. If the protocol is not violated, standard .java files are produced for every .mungo file in the package.  The resulting code can then be ran as any standard Java code.

Typestate is a refinement of the concept of a type in programming [7, 8]. Typestate is the con- straining of a sequence of calls. In programming certain methods can sometimes only be called in certain states. Typestate checks what method is available to be called in which state and what the subsequence state of each method is. Furthermore Typestate can check at compile time that specifications are followed. Through Typestate analysis, it is possible to track the degree of initialisation of variables, guaranteeing that operations would never be applied on improperly initialised data.

The following work was undertaken on the Mungo tool:
\begin{itemize}
  \item the tool has been moved from the Java 1.4 compiler to the Java 1.8 compiler and adapted to work with the new framework.(joint work with Dr. Dimitrios Kouzapas)
  \item moving the tool to a new compiler framework was a good opportunity for some refactoring(such as getting rid of dead code or method extraction).
  \item the tool has been extended to support Java enumerations.(joint work with Dr. Dimitrios Kouzapas)
  \item updated repositories
  \item work has been undertaken to support generic types
  \item work has been undertaken to typecheck exceptions, in a simple form at the moment
\end{itemize}


Areas of future work:
\begin{itemize}
  \item aliasing
  \item typecheck generic types
  \item support annotations
\end{itemize}

\subsection{StMungo}
\label{sub:StMungo}

StMungo (Scribble to Mungo)\footnote{http://www.dcs.gla.ac.uk/research/mungo/} is a Java-based tool, developed by Dr. Ornela Dardha at the University of Glasgow, that translates Scribble\footnote{http://www.scribble.org} local protocol specifications into Mungo specifications.

Scribble is a language that can describe application-level protocols among communicating systems. A protocol represents an agreement on how participating entities interact with each other.

After the Scribble protocol is translated to a Mungo specification, Mungo\ref{Mungo} is used to generate a Java implementation for the protocol. This tool allows an easy transition from a Scribble global protocol definition to working Java implementation. We start by specifying distributed multiparty protocol in Scribble. We can then use the Scribble tool chain to validate and project the global protocol into a local one describing the interactions from the point of view of a specific participant. For every Scribble local protocol, StMungo will produce .mungo files containing: a typestate specification describing the local protocol as a sequence of method calls, an API for the participant implementing the typestate methods and a main class skeleton calling the methods in the typestate.

To improve this tool various extensions, refactoring has been undertaken:

\begin{itemize}
  \item extended the tool to translate messages with no payload i.e.
  \begin{lstlisting}[basicstyle=\footnotesize]
    message_operator ()
  \end{lstlisting}
  \item extended the tool to translate messages with multiple payload i.e.
  \begin{lstlisting}[basicstyle=\footnotesize]
    message_operator ( payload_type1, ..., payload_typen )
  \end{lstlisting}
  \item extended the tool to translate messages without a message signature i.e.
  \begin{lstlisting}[basicstyle=\footnotesize]
    ( payload_type1, ..., payload_typen )
  \end{lstlisting}
  \item extended the tool to translate messages with annotated payloads i.e.
  \begin{lstlisting}[basicstyle=\footnotesize]
    message_operator ( annotation:payload_type)
  \end{lstlisting}
  \item various small improvements to allow most translations to run without having to be edited by a human
  \item various improvements allowing the tool to crash gracefully
  \item adapted the tool to work with multiple versions of scribble specification
  \item improved the tool by implementing support for special cases of recursions nested in choice structures
  A simple example of a problematic scribble specification is:
  \begin{lstlisting}[basicstyle=\footnotesize]
    global protocol Example(role S, role C) {
    choice at C{
            rcpt(String) from C to S;
        } or {
            msg(String) from C to S;
            rec loop {
                    subject(String) from C to S;
                    continue loop;
                }
            }
        }
    }
  \end{lstlisting}
  \item improved the tool by implementing support for special cases of nested choice inside a recursion

  A simple example of a problematic scribble specification is:
  \begin{lstlisting}[basicstyle=\footnotesize]
  global protocol ProtocolName(role S, role C) {
    command(String) from C to S;
    rec overall {
    choice at S
    	{
    	ok(String) from S to C;
    			choice at S {end(String) from S to C;}
    			or {
    			sum(String) from S to C;
    			}
    			message(String) from S to C;
    	} or {	 error(String) from S to C;	}
    	continue overall;
    }
    }
  \end{lstlisting}

  \item extending the tool to translate to support inlined protocols and sub-protocols(ongoing)
  \item kept it up to date with changes in Mungo
  \item refactoring(such as method extraction or getting rid of dead code) to keep everything simple
  \item regression testing to find any new bugs introduced
\end{itemize}

Areas of future work:
\begin{itemize}
  \item test harness
  \item extension to translate more complex constructs such as interruptible or parallel
  \item keeping it up to date with changes in Scribble and Mungo
\end{itemize}


\subsection{Usecases}
\label{sub:usecases}

To better understand the expressive power of current session type technology together with any limitations that may need to be addressed, the current use case repository\footnote{https://github.com/epsrc-abcd/session-types-use-cases} was surveyed as a first step. As a second step, new real-world examples were sought. From the various protocols looked after, representations were attempted for two, Paxos and the File Transfer Protocol(FTP).

The first protocol chosen as a usecase was Paxos. Paxos is a protocol for solving consensus in a network of unreliable processes. It ensures that a single value among the proposed values can be chosen. It assumes an asynchronous, non-Byzantine model.\cite{paxos}

Two major advantages of Paxos are that it is provably correct in asynchronous networks that eventually become synchronous and it does not block if a majority of participants are available.Furthermore it has provably minimal message delays in the best case.
Despite it's reputation of being difficult to understand \& implement it is widely, a couple of examples would be Google in Chubby, Yahoo use something based on it in ZooKeeper.
The protocol comes with three roles and a two-phase approach. A proposer
responsible for initiating the protocol, that handles client requests and
proposes values to be chosen. An acceptor that responds to messages from proposers by either rejecting them or agreeing in principle and making a promise about the proposals it will accept in the future. An a listener or learner, who
wants to know which value was chosen. Each Paxos server can act as any or all 3 roles.

% Concepts
% proposalNumber: used to order the proposals so that acceptors can agree on which proposals came before or after
% acceptedProposal: the number of the last proposal the server has accepted
% acceptedValue: the value from the most recent proposal the server has accepted
% minProposal: the number of the smallest proposal this server will accept
%
%
% 2-phase approach
% Phase 1: broadcast Prepare
% find out about any chosen values
% block older proposals that have not yet completed
% Phase 2: broadcast Accept
% ask acceptors to accept a specific value
% acceptors can reply with reject or accept
% Consensus is reached when a majority of the nodes have accepted; and the coordinator broadcasts a commit message to all nodes.

%
%  A node is elected to be a Proposer
% The Proposer selects a value and sends it to all nodes (called Acceptors in Paxos) in an accept-request message.
% Acceptors can reply with reject or accept
% Consensus is reached when a majority of the nodes have accepted; and the coordinator broadcasts a commit message to all nodes.
%
% Algorithm
% Proposer side
% 1. Choose new proposal number n
% 2. Broadcast Prepare(n, value) to all Acceptors
% 4. When responses received from the majority
% If any acceptedValue returned, replace internalValue with acceptedValue for highest acceptedProposal
% 5. Broadcast Accept(n, value) to all servers
% 7. When responses received from majority:
% any rejections(result > n)? goto 1
% Otherwise, the value is chosen
% Acceptor side
% 3. Respond to Prepare(n, value)
% if n > minProposal then minProposal = n
% Return(acceptedProposal, acceptedValue)
% 6. Respond to Accept(n, value):
% if n>= minProposal then acceptedProposal = minProposal = n acceptedValue = value
% Return(minProposal)
% Acceptors must record minProposal, acceptedProposal, and acceptedValue on stable storage(disk)
%
% each Paxos server can contains both components

Some representations of the algorithm have been attempted using Scribble, combined with StMungo and Mungo to give a working Java code. However in trying to represent it some shortcomings of the toolset became apparent:
\begin{itemize}
  \item representing broadcasting
  \item representing quorum/a majority
  \item representing express the dynamic aspects such as processes failing, restarting
  \item express multiple instances of the protocol
\end{itemize}

The work on Paxos has been presented during the last ABCD meeting in January 2016. Work on a better Paxos representation has been paused for the moment, other tasks taking priority such as improving the tools.


The second protocol chosen was the File Transfer Protocol described by Request for Comments(RFC): 959\footnote{https://tools.ietf.org/html/rfc959}. FTP is a standard network protocol for transferring  files between a client and server on a network.  FTP is an unusual protocol in that it utilizes two ports, a data port and a command(control) port. FTP may run in active or passive mode, which determines how the data connection is established. In both cases, the client creates a TCP control connection from a random, usually an unprivileged, port number to the FTP server command port 21.
%
% In active mode, the client starts listening for incoming data connections from the server on a port {\it port\_number}. It sends the FTP command PORT {\it  port\_number} to inform the server on which port it is listening. The server then initiates a data channel to the client from its port 20, the FTP server data port.
%
% In situations where the client is behind a firewall and unable to accept incoming TCP connections, passive mode may be used. In this mode, the client uses the control connection to send a PASV command to the server and then receives a server IP address and server port number from the server, which the client then uses to open a data connection from an arbitrary client port to the server IP address and server port number received.
Some representations of the algorithm have been attempted using Scribble, combined with StMungo and Mungo to give a working Java code. However in trying to represent it some shortcomings of StMungo became apparent. Hence, work on an FTP representation has been paused to improve StMungo and Mungo, to allow a better representation. 

Work on both usecases is planned to be restarted in the autumn, and make the base for a paper later on in the year. 

\subsection{User study}
\label{us}


% Session types good for programmers. Will like/use/find helpful in their day to day programming.

% This phd aims to explore how programmers can interact with session types, if they find them useful and in which ways. Make this intrecation more meaningful.

New programming language constructs are more often than not introduced without first exploring how well suited their are for their purpose or how they would be used in the real world. While proving they solve the problem is a good thing, checking how well they solve it would be nice.
To explore how well programmers would interact with session types an initial study has been devised. This can be found in appendix \ref{usertrial}. It is intended to explore whether programmers can learn what session types are, identify how they should be used, and reason about session typed code correctly. At this stage there are little constraints on the type of participant. Since one of the claims behind session types is that they make reasoning about communication easier, participants with less programming experience are welcome. It would be interesting to see if there are any differences and what differences there are between participants of different backgrounds. The only requirement in this case being a working knowledge of Java.

Difficulties: choosing meaningful for the user to carry out, deciding what constitutes a good understanding of session types.


% Session types are formal descriptions of communication protocols.
% The formal nature of session types enables their incorporation
% into programming languages and tools so they can be used
% to drive the software development process.
% %which can be incorporated into programming languages and tools
% %so that typechecking can be used to verify the dynamic structure
% %of communication as well as the static structure of data.
% Research on session types has moved from theoretical studies
% towards practical applications, and the field is maturing to the
% point where session types can be integrated into software
% engineering methodologies.
% However, most of the literature is not sufficiently accessible to a
% wider academic and industrial audience;
% more systematic technology transfer is required.
% The present paper provides an entry point to a substantial repository
% of practical use-cases for session types in a range of application domains.
% It aims to be an accessible introduction to session types as a practical
% foundation for the development of communication-oriented software, and to
% encourage the adoption of session-type-based tools.


% Regardless, assessing the utility of typestate through experimentation is likely to be essential in convincing those outwith the research community that it has value, and should be included in contemporary programming languages.


% In order to evaluate the practicality of the Hanoi language for typestate modelling, an experimental design was desired with the aim of evaluating the following research questions:
% • Can programmers reason effectively about the semantics of Hanoi?
% • Is the effectiveness with which a programmer can reason about a Hanoi model influ- enced by the presentation of the model?
% 5.1. Experimentsconsidered 117
%  • Will programmers prefer either the DSL or annotation model types?
% These questions should be evaluated in the context of one or more of the following tasks:
% • Deciding whether a Hanoi model is semantically valid.
% • Writing code that does not violate constraints in a Hanoi model.
% • Producing a semantically valid Hanoi model against an informal specification. • Identifying whether code violates constraints in a Hanoi model.
% Different experimental designs were considered, oriented around questions related to these tasks, before settling on a final experimental design.


\section{Other Activities}
\label{sec:Activities}

As part of the first year of my PhD, various additional activities have been undertaken, such as training courses(detailed in appendix\ref{traininglog}), ABCD group meetings of various sizes, seminars and talks(e.g. Tht Scottish programming language seminar series, FATA seminars) which improved my knowledge of the field and gave me some insight of the exciting research ongoing, as well as my own ignorance. Some notable events attended were the BETTY (Behavioural Types for Reliable Large-Scale Software Systems)\footnote{http://www.behavioural-types.eu/meetings/wg-mc-meetings-17th-18th-march-2016-in-malta} meeting in March, and Wadlerfest\footnote{http://events.inf.ed.ac.uk/wf2016/}, held in celebration of Philip Wadler's 60th birthday, or LFCS30\footnote{http://events.inf.ed.ac.uk/lfcs30/}, to celebrate the 30th anniversary of the founding of the LFCS.


\section{Future Work}
\label{sec:future}

Some of the future work planned has been highlighted throughout this report. The activities planed for next year fall into several interconnected themes:
\begin{itemize}
\item Language usability and evaluation. As mentioned above in \ref{us} a user study is in plan for the near future, its results to be analysed and written up and submitted to PLATEAU 2016\footnote{http://2016.splashcon.org/track/plateau2016}. A second more focused study is also in plan for the later part of the year. 
\item Tool development. Some of the work that is to be done has already been discussed earlier, however it is expected that the need for various different improvements will arise. Thus this will be an ongoing activity through the year. 
\item Training and development. This category contains any training courses required by the school as well as any academic events to be attended. Over the summer, I will attend BETTY (Behavioural Types for Reliable Large-Scale Software Systems) in Limassol, Cyprus. The summer school will cover closely related areas of interest such as multiparty session types, linear logic, Practical programming with session types. In September I will help with the PPDP, LOPSTR and SAS conferences in Edinburgh.
\item Refining initial research objectives and questions. This is really the miscellaneous category, containing work to be continued such as the usecases\ref{usecases}, and any new ideas that will arise from the user study.
\end{itemize}

For a detailed plan, a Ganntt chart is available in appendix \ref{appendix:gantt}.

