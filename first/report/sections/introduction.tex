\section{Introduction}


Nowadays, distributed systems are ubiquitous and communication is an important feature and reason for its success. Communication-centred programming has proven to be one of the most successful attempts to replace shared memory for building concurrent, distributed systems. Communication is easier to reason about and scales well as opposed to shared memory, making it a more suitable approach for systems where scalability is a must, as in the case of multi-core programming, service-oriented applications or cloud computing\cite{abcd}.
%make this one flow better

Communication is usually standardised via protocols that specify the possible interactions between the communicating parties in a specific order. Mainstream programming languages fail to adequately support the development of communication-centred software. Thus, implementations of communication behaviours are based on informal protocol specification and thus informal verification. As a result they are prone to errors such as communication mismatch, when the message sent by one party is not expected by the other party, or deadlock, when two parties are waiting for a message from each other causing the system to block\cite{abcd}. 

To allow formal protocol specification within the programming language session types have been devised. Session types describe communication by specifying the type and direction of messages exchanged between two parties[6]. Programmers can express a protocol specification as a session type, which can guarantee, within the scope where the session type applies, that communications will always match and the system will never deadlock.
The main goal of the ABCD project\cite{abcd} is to improve the practice of software development for concurrent and distributed systems through the use of session types. This is to be accomplished through built-in language support for protocol codification in existing languages such as Java or Python, in new languages such as Links\cite{abcd}, inter-language interoperability via session types, and through adapting interactive development environments and modelling techniques to support session types. Logical and automata foundations of session types will be further developed to express a wider class of behaviour and, as need arises, to support the former. Empirical studies to assess methodologies and tools are to be carried out, with results being used to improve language and tool design and implementation.

\subsection{Research Questions}

This PhD will attempt to answer the following questions considering both the theoretical and practical aspects of session types.

\begin{itemize} 
\item How can programmers use session types in meaningful ways?
\begin{itemize}
\item What are the strengths and weaknesses of the options for programming with session types?
\item Can programmers reason effectively about communication using session types?
\item Does the method of expressing session types influence the programmers' the ability to reason effectively about the system?
\end{itemize}
These questions will be explored mainly trough user studies, the first of which is outlined below in \ref{us}.

\item What are the theoretical and practical challenges to expressing “real” life protocols with session types? 
How does the theory need to be extended to help overcome these? How do existing tools need to be improved? Is there a need for additional tools?
\end{itemize}
\subsection{Thesis Statement}

Session types are a useful addition to the syntax and semantics of modern languages. Programming with session types and the constraints that this comes with i.e. linearity can be understood and used by real world programmers. Moreover session types can help programmers understand the problem they are tying to solve with more ease and structure their code better.
Session typed languages provide useful additional safeguards and diagnostic information that lead to a system with the expected behaviour with less effort.

% \subsection{Outline}
\subsection{Outline}

I have spent the first 9 months of my PhD reviewing existing work within the fields of session types, typestate, and language usability and evaluation. A short literature review of the most relevant work can be found in \ref{litreview}. Alongside that, I have done some implementation work on two session type tools developed at the University of Glasgow as part of the ABCD project\cite{abcd}, namely Mungo, and StMungo\cite{mungo}, as well as looked at some possible usecases. This will be discussed in more detail in section \ref{Research}. At present my focus is on defining a suitable methodology and carrying out a user study to explore how "real-world" programmers will interact with session types through Mungo, StMungo and Scribble. More on this can be found in \ref{us}.

Future work along with a plan for the second year is discussed in section \ref{future}.


