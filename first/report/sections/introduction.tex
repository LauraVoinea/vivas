\section{Introduction}


Communication is usually standardised via protocols that specify the possible interactions between the communicating parties in a specific order. Implementations of communication behaviours are based on informal protocol specification and thus informal verification. As a result they are prone to errors such as communication mismatch, when the message sent by one party is not expected by the other party, or deadlock, when two parties are waiting for a message from each other causing the system to block[1]. To allow formal protocol specification within the programming language session types have been devised. Session types describe communication by specifying the type and direction of messages exchanged between two parties[6]. Programmers can express a protocol specification as a session type, which can guarantee, within the scope where the session type applies, that communications will always match and the system will never deadlock.
The main goal of the ABCD project[1] is to improve the practice of software development for concurrent and distributed systems through the use of session types. This is to be accomplished through built-in language support for protocol codification in existing languages such as Java or Python, in new languages such as Links[3], inter-language interoperability via session types, and through adapting interactive development environments and modelling techniques to support session types. Logical and automata foundations of session types will be further developed to express a wider class of behaviour and, as need arises, to support the former. Empirical studies to assess methodologies and tools will be carried out, with results being used to improve language and tool design and implementation.
To help improve the practice of software development and improve developer adoption there is a need for extending integrated development environments(IDEs) and for carrying
out empirical studies to investigate the impact of session types.






Nowadays, distributed systems are ubiquitous and communication is an important feature and reason for its success. Communication-centred programming has proven to be one of the most successful attempts to replace shared memory for building concurrent, distributed systems. Communication is easier to reason about and scales well as opposed to shared memory, making it a more suitable approach for systems where scalability is a must, as in the case of multi-core programming, service-oriented application or, or web services[1].

Mainstream programming languages and methodologies fail to adequately support communication-centred programming. Support is required to exclude insidious programming errors at the development stage; this is key to avoiding the deployment of programs which may be faulty at the execution stage.

% communication is a prominent feature of distributed software systems.

%Building upon IT paradigms such as service-orientation and cloud computing, these infrastructures are rapidly converging into so-called ‘communication - centred’ software systems, that is, systems in which software artefacts rely on communication protocols to achieve their goal





%Recent years have seen a rapid increase in research on behavioural types, driven partly by the need to formalize and codify communication structures as computing moves from the data-processing era to the communication era, and partly by the realization that type-theoretic techniques can provide insight into the fine structure of computation.

\section{Thesis}

\subsection{Thesis Statement}

Session types are a useful addition to the syntax and semantics of modern languages. Programming with session types and the constraints that this comes with i.e. linearity can be understood and used by real world programmers. Moreover session types can help programmers understand the problem they are tying to solve with more ease and structure their code better.
Session typed languages provides useful safeguards and diagnostic information that lead to a correct program with less effort.

communication-based software


\subsection{Thesis Outline}

communication-based software everywhere!! aaaa!



is the additional complexity justifiable? Can typestate be reasoned about effectively by “real” programmers?
