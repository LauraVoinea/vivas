\section{Thesis}
%The research is done in the context of the From Data Types to Session Types: A Basis for Concur- rency and Distribution (ABCD) project, an EPSRC programme grant (EPSRC EP/K034413/1). The ABCD project aims to improve the practice of software development for concurrent and dis- tributed systems through the use of session types. Session types describe communication by speci- fying the type and direction of messages exchanged between two parties. Programmers can express a protocol specification as a session type, which can guarantee, within the scope where the session type applies, that communications will always match and the system will never deadlock.

%0.2 Aims and objectives
%The aim of this research is twofold, first to explore how session types can be integrated in interactive development environments (IDEs), second, to evaluate programming language designs, implementa- tion, and tools developed as part of the ABCD project in relation to relevant industry case studies. To achieve this, tools and theory developed as part of the ABCD project for session types in Java will be enhanced to allow programmers to easily use them.
%Empirical studies will be used to explore the extent to which session types can be used to provide guidance to developers, and gather data that will help to continuously improve language de- signs and implementation. Evaluating programming languages and constructs is a relatively young field with much work still to be done. As a result, this research will investigate which empirical methods, techniques and software design metrics would be most appropriate to evaluate language designs that support session types. Furthermore, new tools will be implemented to measure de- veloper productivity when modelling communication using session types, and to explore if session types aid programmers in building more reliable concurrent and distributed systems.

%0.3 Potential applications and benefits
%It is expected that this research will have a significant impact on the tools developed as part of the ABCD project, on the theory behind them, and ultimately on the industry. Empirical studies will gather data that will help to continuously improve language designs and implementation to something that the industry can readily use. The tools developed and further developed will help automate the overall process of programming with session types, which is in the best interest of adoption. Furthermore, tools can assist in understanding how a language works, and can thus help programmers better understand and work with session types. In turn programmer efficiency as well as the program’s performance and reliability will be improved.
%This research will also make a substantial contribution to the field of evaluation and usability of programming languages and tools by further developing methodologies behind language and tool evaluation and tools to support evaluating programming languages.


Session types good for programmers. Will like/use/find helpful in their day to day programming.

This phd aims to explore how programmers can intract with session types, if they find them useful and in which ways. Make this intrecation more meaningful. 
